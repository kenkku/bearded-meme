% --- Template for thesis / report with tktltiki2 class ---
% 
% last updated 2013/02/15 for tkltiki2 v1.02


\documentclass[grading, english, finnish]{tktltiki2}

% tktltiki2 automatically loads babel, so you can simply
% give the language parameter (e.g. finnish, swedish, english, british) as
% a parameter for the class: \documentclass[finnish]{tktltiki2}.
% The information on title and abstract is generated automatically depending on
% the language, see below if you need to change any of these manually.
% 
% Class options:
% - grading                 -- Print labels for grading information on the front page.
% - disablelastpagecounter  -- Disables the automatic generation of page number information
%                              in the abstract. See also \numberofpagesinformation{} command below.
%
% The class also respects the following options of article class:
%   10pt, 11pt, 12pt, final, draft, oneside, twoside,
%   openright, openany, onecolumn, twocolumn, leqno, fleqn
%
% The default font size is 11pt. The paper size used is A4, other sizes are not supported.
%
% rubber: module pdftex

% --- General packages ---

\usepackage[utf8]{inputenc}
\usepackage[T1]{fontenc}
\usepackage{lmodern}
\usepackage{microtype}
\usepackage{amsfonts,amsmath,amssymb,amsthm,booktabs,color,enumitem,graphicx}
\usepackage[pdftex,hidelinks]{hyperref}

\usepackage{listings}
\lstset{
    language=Haskell,
    basicstyle=\small
}

% Automatically set the PDF metadata fields
\makeatletter
\AtBeginDocument{\hypersetup{pdftitle = {\@title}, pdfauthor = {\@author}}}
\makeatother

% --- Language-related settings ---
%
% these should be modified according to your language

% babelbib for non-english bibliography using bibtex
\usepackage[fixlanguage]{babelbib}
\selectbiblanguage{finnish}

% add bibliography to the table of contents
\usepackage[nottoc]{tocbibind}
% tocbibind renames the bibliography, use the following to change it back
\settocbibname{Lähteet}

% --- Theorem environment definitions ---

\newtheorem{lau}{Lause}
\newtheorem{lem}[lau]{Lemma}
\newtheorem{kor}[lau]{Korollaari}

\theoremstyle{definition}
\newtheorem{maar}[lau]{Määritelmä}
\newtheorem{ong}{Ongelma}
\newtheorem{alg}[lau]{Algoritmi}
\newtheorem{esim}[lau]{Esimerkki}

\theoremstyle{remark}
\newtheorem*{huom}{Huomautus}


% --- tktltiki2 options ---
%
% The following commands define the information used to generate title and
% abstract pages. The following entries should be always specified:

\title{Haskell}
\author{Tero Keinänen}
\date{\today}
\level{~}
% \abstract{Tiivistelmä.}

% The following can be used to specify keywords and classification of the paper:

% \keywords{avainsana 1, avainsana 2, avainsana 3}

% classification according to ACM Computing Classification System (http://www.acm.org/about/class/)
% This is probably mostly relevant for computer scientists
% uncomment the following; contents of \classification will be printed under the abstract with a title
% "ACM Computing Classification System (CCS):"
% \classification{}

% If the automatic page number counting is not working as desired in your case,
% uncomment the following to manually set the number of pages displayed in the abstract page:
%
% \numberofpagesinformation{16 sivua + 10 sivua liitteissä}
%

\linespread{1.5}

\begin{document}

% --- Front matter ---

\frontmatter      % roman page numbering for front matter

\maketitle        % title page
% \makeabstract     % abstract page

\tableofcontents  % table of contents

% --- Main matter ---

\mainmatter       % clear page, start arabic page numbering

\section{Johdanto}

Haskell on puhtaasti funktionaalinen ohjelmointikieli. Puhtaasti funktionaalisissa ohjelmointikielissä funktiot ovat matemaattisia kuvauksia~\cite[s.~1]{Sab98}. Kuvauksen määrittelyjoukko vastaa puhtaan funktion sallittujen todellisten parametrien joukkoa. Maalijoukko puolestaan vastaa funktion mahdollisten palautusarvojen joukkoa. Puhdas funktio siis palauttaa aina saman arvon samoilla parametreilla. Puhtaus myös tarkoittaa, että funktioilla ei voi olla sivuvaikutuksia. Käytännössä jokin tapa aiheuttaa sivuvaikutuksia on kuitenkin yleiskäyttöisessä ohjelmointikielessä pakollinen, sillä ilman sivuvaikutuksia ei esimerkiksi tiedostojen lukeminen ja kirjoittaminen onnistu. Koska lausekkeiden evaluointijärjestystä ei ole määritelty tarkasti ja se voi vaihdella ohjelmointikielen implementaatioiden välillä, käytetään Haskellissa sivuvaikutuksien integroimiseksi erityistä tietotyyppiä, monadia~\cite[luku~7]{Mar10}. Sivuvaikutuksia sisältäville tapahtumille voidaan määritellä yksiselitteinen järjestys käyttämällä monadeja. Monadit ovat hyödyllisiä myös muihin tarkoituksiin, ja niitä on käytetty Haskellissa esimerkiksi jäsentäjien toteuttamisessa.

Toinen Haskellin perusominaisuuksista on laiska evaluointi. Laiskasti evaluoituvassa ohjelmointikielessä lausekkeen arvo lasketaan vasta tarvittaessa (\emph{call-by-need})~\cite[s.~12-8]{Hud07}. Tämä mahdollistaa esimerkiksi Haskellissa usein käytetyt näennäisesti äärettömät listat. Laiskan evaluoinnin valitseminen johti myös puhtauteen. Kun evaluointi etenee riippuen siitä mitä arvoa tarvitaan seuraavaksi, on sivuvaikutuksia vaikea suorittaa funktiokutsun yhteydessä luotettavasti~\cite[s.~12-8]{Hud07}. Osa funktioista saattaa jäädä evaluoimatta, jos niiden palauttamia arvoja ei tarvittukaan. Evaluointi saattaa myös tapahtua eri järjestyksessä kuin ohjelmoija on olettanut.

Haskell on vahvasti ja staattisesti tyypitetty. Käytössä on Hindley--Milner-tyyppijärjestelmä~\cite[luku~4.1]{Mar10}, joka sisältää tyyppipäättelyalgoritmin. Lausekkeen tyypin määrittely on vapaaehtoista ja tarpeen lähinnä, kun kääntäjä ei pysty tyyppiä päättelemään tai jos tyyppi halutaan näkyville dokumentointitarkoituksessa. Hindley--Milner-tyyppijärjestelmää on Haskellissa laajennettu tyyppiluokilla, jotka mahdollistavat funktioiden kuormittamisen~\cite[luku~4.1]{Mar10}.

Ajatus Haskellista syntyi \foreignlanguage{english}{Functional Programming Languages and Computer Architecture} -konferenssissa vuonna 1987, jossa päätettiin muodostaa komitea uuden puhtaasti funktionaalisen, laiskasti evaluoidun ohjelmointikielen suunnittelemiseksi~\cite[s.~12-1]{Hud07}. Nimen valitseminen katsottiin tärkeäksi ja heti ensimmäisessä komitean tapaamisessa uusi kieli päätettiin nimetä matemaatikko Haskell Curryn mukaan~\cite[s.~12-4]{Hud07}.  Ensimmäinen versio Haskell 1.0 ilmestyi vuonna 1990~\cite{HW90} ja kielen uusin versio on vuodelta 2010~\cite{Mar10}.


\section{Kielen esittely}

Haskell on funktionaalinen ohjelmointikieli, joten se perustuu funktioiden määrittelemiseen. Tyypillinen esimerkki, kertomafunktio, näyttää Haskellissa seuraavalta:
\begin{lstlisting}
kertoma :: Integer -> Integer   
kertoma 0 = 1                 
kertoma n = n * kertoma (n - 1)
\end{lstlisting}
Ensimmäisellä rivillä määritellään \lstinline|kertoma|-funktion tyypiksi \lstinline|Integer -> Integer|. Lausekkeen tyypin esittäminen tässä yhteydessä ei ole välttämätöntä, sillä kääntäjä pystyy sen myös päättelemään. Funktioille on kuitenkin tapana määritellä tyyppi, sillä se esimerkiksi helpottaa dokumentaation automaattista tuottamista. Toisella ja kolmannella rivillä on funktion osittaiset määritelmät. Tämä määritelmä käyttää hyväksi hahmonsovitusta, jossa lauseketta evaluoitaessa todelliset parametrit yritetään sovittaa määritelmiin ylhäältä alaspäin kokeillen.

Tarkastellaan toisena esimerkkinä funktionaalisessa ohjelmoinnissa usein käytettyä map-funktiota:
\begin{lstlisting}
map :: (a -> b) -> [a] -> [b]
map _ []     = []
map f (x:xs) = f x : map f xs
\end{lstlisting}
Map on siis funktio, jonka parametreja ovat muunnosfunktio (\lstinline|a -> b|) ja lista, jolle muunnos suoritetaan (\lstinline|[a]|). Map ottaa listan, joka sisältää alkioita tyyppiä \lstinline|a|, suorittaa jokaiselle muunnosfunktion, ja palauttaa listan muunnettuja arvoja (\lstinline|[b]|). Ensimmäisenä parametrina olevan funktion tyyppi merkitään sulkuihin, jotta se ymmärretään yhdeksi parametriksi. Haskellissa funktiot siis ovat arvoja, joita voidaan antaa toisille funktioille parametreina. Kolmannella rivillä käytetään hyväksi listan destrukturointia ensimmäiseen alkioon (x) ja loppuosaan (xs). Haskellin sisäänrakennettuja listoja luodaan :-operaattorilla sekä tyhjää listaa tarkoittavalla \lstinline|[]|-literaalilla. Listan alkiot liitetään yhteen listan konstruktorilla, :-operaattorilla, jonka ensimmäinen parametri on listan ensimmäinen alkio ja toinen parametri listan loppuosa. Esimerkiksi lista kokonaislukuja näyttäisi seuraavalta:
\begin{lstlisting}
1 : (2 : (3 : (4 : [])))
\end{lstlisting}
Sulut ovat kuitenkin tarpeettomat, joten listan voi merkitä myös näin:
\begin{lstlisting}
1 : 2 : 3 : 4 : []
\end{lstlisting}
Listoille löytyy myös erillinen, hiukan kätevämpi syntaksi:
\begin{lstlisting}
[1, 2, 3, 4]
\end{lstlisting}

Funktioiden soveltaminen Haskellissa tapahtuu oletuksena postfix-merkinnällä, kuten map-funktion määritelmässä funktioiden f ja map soveltaminen, mutta myös infix-muoto on mahdollinen. Map-funktiossa listan kontruktori (:) on infix-muodossa.



% --- References ---
%
% bibtex is used to generate the bibliography. The babplain style
% will generate numeric references (e.g. [1]) appropriate for theoretical
% computer science. If you need alphanumeric references (e.g [Tur90]), use
%
% \bibliographystyle{babalpha-lf}
%
% instead.

\newpage

\bibliographystyle{babalpha-lf}
\bibliography{references-fi}


% --- Appendices ---

% uncomment the following

% \newpage
% \appendix
% 
% \section{Esimerkkiliite}

\end{document}
